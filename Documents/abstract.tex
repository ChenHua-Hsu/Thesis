%----------------------------------------------------------------------------------------
%	ABSTRACT in Chinese
%----------------------------------------------------------------------------------------

\begin{CJK}{UTF8}{bkai}
    ~ \\
    ~ \\
    ~ \\
    \huge \textbf{中文摘要}
\end{CJK}

\vspace{1.5cm}

\begin{CJK}{UTF8}{bkai}
    隨著對撞機的不斷擴建和升級,物理學家面臨著越來越複雜的實驗需求,這導致對計算資源的需求急劇增加。現有的計算能力將難以持續支撐Geant4軟體完成精確且大規模的全套物理計算模擬,因此,尋求一種更加高效、快速的模擬方法已成為當前的研究重點。在此論文中,我們提出了使用擴散模型作為核心演算法,並結合transformer模型,嘗試模擬粒子能量在探測器內部的空間分佈。這一方法不僅能夠顯著加速模擬過程,還保持了與Geant4模擬結果相似的精度。本研究的最大特色在於其能夠生成與Geant4預測高度一致的三維能量分佈圖,而不僅僅是如同大多數類似研究所展示的在一維空間上的能量分佈。

\end{CJK}

\vspace{0.5cm}

\begin{CJK}{UTF8}{bkai}
    \textbf{關鍵詞}:快速模擬、擴散模型、Transformer、CaloChallenge、HGCal。
\end{CJK}

\linespread{0}

\begin{CJK}{UTF8}{bkai}
    \addchaptertocentry{中文摘要}
    \clearpage
\end{CJK}

%----------------------------------------------------------------------------------------
%	ABSTRACT in English
%----------------------------------------------------------------------------------------

\begin{abstract}
    \addchaptertocentry{\abstractname} % Add the abstract to the table of contents

    As particle colliders continue to expand and upgrade, physicists face increasingly complex experimental demands, which in turn have led to a sharp rise in the need for computational resources. The current computational power will struggle to support full-scale and precise simulations using Geant4 software, especially as the scale of experiments grows. Therefore, finding a more efficient and fast simulation method has become a pressing priority in current research. In this thesis, we propose using a diffusion model as the core algorithm, coupled with a transformer model, to simulate the spatial distribution of particle energy within the detector. This approach not only significantly accelerates the simulation process but also maintains a level of accuracy comparable to Geant4 simulations. The key feature of this research lies in its ability to generate three-dimensional energy distributions that closely match those predicted by Geant4, rather than the one-dimensional energy distributions typical of most similar studies.

    \vspace{0.5cm}
    \textbf{Keywords}: Fast Simulation, Diffusion Model, Transformer, CaloChallenge, HGCal.
\end{abstract}
