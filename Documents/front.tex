%----------------------------------------------------------------------------------------
%	Title page (Main title, hardcover version)
%----------------------------------------------------------------------------------------

\thispagestyle{empty}

\begin{titlepage}

\parskip=8pt					%space btw paragraphs
\linespread{1.5}

\begin{center}

\thispagestyle{empty}

\begin{center}
\begin{CJK}{UTF8}{bkai}
 \vspace {4.0cm}
\LARGE \noindent   國立臺灣大學 理學院物理學研究所 \\
        \vspace {0.0cm}
        碩士論文\\
            \end{CJK}
            \vspace {0.2cm}
{
    \noindent
        \large Department of Physics\\
        \vspace {0.0cm}
    \large College of Science\\
        \vspace {0.0cm}
    \large National Taiwan University\\
        \vspace {0.0cm}
    \large Master's Thesis\\
}

\vspace {2.0cm}
\setlength{\baselineskip}{32pt}
\begin{CJK}{UTF8}{bkai}
\LARGE  \noindent
% 在KOTO實驗利用集群觸發系統\\研究$K_L^0 \to \pi^0 \nu \bar{\nu}$與$K_L^0 \to \pi^0 \gamma \gamma$
通過基於評分的擴散模型實現快速HGCal探測器模擬
\end{CJK}

\vspace {0.5cm}
\setlength{\baselineskip}{24pt}
{\LARGE \noindent Fast HGCal Detector Simulation via Score-Based Diffusion Models
}\\

\vspace {5.0cm}

\begin{CJK}{UTF8}{bkai}
\Large \noindent   徐振華\\
\vspace {0.0cm}
\Large Chen-Hua Hsu\\
\vspace {0.2cm}
\Large 指導教授:陳凱風 教授\\
\vspace {0.0cm}
\Large Advisor: Kai-Feng Chen, Ph.D.\\

\vspace {0.4cm}
\Large \noindent   中華民國113年10月\\
\Large \noindent   October 2024\\
\end{CJK}

\end{center}

\end{center}

\end{titlepage}

\thispagestyle{empty}
~\\
\cleardoublepage

%----------------------------------------------------------------------------------------
%	Title page (2nd title page)
%----------------------------------------------------------------------------------------

\thispagestyle{empty}

\begin{titlepage}
\begin{center}

\vspace*{.06\textheight}
{\scshape\LARGE \univname\par}\vspace{1.5cm} % University name
\textsc{\Large Master's Thesis}\\[0.5cm] % Thesis type

\HRule \\[0.4cm] % Horizontal line
% {\huge \bfseries {Study of $K_L^0\to\pi^0\nu\overline{\nu}$ and $K_L^0\to\pi^0\gamma\gamma$ \\ with the Cluster-Finding Trigger at KOTO} \par}\vspace{0.4cm} % Thesis title
{\huge \bfseries {Fast HGCal Detector Simulation via Score-Based Diffusion Models} \par}\vspace{0.4cm} % Thesis title
\HRule \\[1.5cm] % Horizontal line

\begin{minipage}[t]{0.4\textwidth}
\begin{flushleft} \Large
\emph{Author:}\\
\href{https://github.com/ChenHua-Hsu}{\authorname} % Author name - remove the \href bracket to remove the link
%\authorname
\end{flushleft}

\end{minipage}
\begin{minipage}[t]{0.4\textwidth}
\begin{flushright} \Large
\emph{Supervisor:} \\
%\supname
\href{https://www.phys.ntu.edu.tw/kfjack.html}{\supname} % Supervisor name - remove the \href bracket to remove the link
\end{flushright}
\end{minipage}\\[3cm]

%\vspace{4 cm}

\begin{center}
\includegraphics[scale=0.5]{NTU.jpg}
\end{center}

\vspace{1 cm}

{\Large \today}\\[4cm] % Research group name and department name

\vfill

\end{center}
\end{titlepage}
\linespread{0}

% copyright note
\clearpage
\vspace*{130mm}
\begin{center}
{% begin group
   \thispagestyle{empty}
   %\footnotesize\itshape
   \setlength{\parskip}{\baselineskip}
   \setlength{\parindent}{0pt}

   \copyright\,2024, by Chen-Hua Hsu\\
   ken91021615@hep1.phys.ntu.edu.tw \\

   %hep1 QRcode
   %\includegraphics[width=2cm]{Figures/ORCID.png}

   ALL RIGHTS RESERVED
}% end group
\end{center}


\clearpage
